\emph{Demostración:}
\begin{itemize}
\item[\textcolor{red}{$\bf{a}$.}] Note que queremos optimizar la función:
$$
\ell(\boldsymbol{\mu})=-\frac{np}{2}\log(2\pi)-\frac{n}{2} \log |\boldsymbol{\Sigma}|-\frac{1}{2} \operatorname{tr}\left( \boldsymbol{\Sigma}^{-1}\suma \parcurvo{
\bf{x}_i - \bf{\mu}
}\parcurvo{
\bf{x}_i - \bf{\mu}
}^\top\right)
$$
Sujeto a:
$$
\bf{A}\bf{\mu} = \bf{a} \ssi \bf{\mu}^\top \bf{A}^T = \bf{a}^T
$$
Por tanto para encontrar el óptimo debemos optimizar mediante multiplicadores de Lagrange. Definamos el Lagrangiano:
$$
\mathcal{L}(\bf{\mu}) = -\frac{np}{2}\log(2\pi)-\frac{n}{2} \log |\boldsymbol{\Sigma}|-\frac{1}{2} \operatorname{tr}\left( \boldsymbol{\Sigma}^{-1}\suma \parcurvo{
\bf{x}_i - \bf{\mu}
}\parcurvo{
\bf{x}_i - \bf{\mu}
}^\top\right) + \lambda \parcerrado{
 \bf{\mu}^\top \bf{A}^T 
}
$$
Luego:
\begin{align*}
\de \mathcal{L}(\bf{\mu}) &=
n (\de \bf{\mu})^\top 
    \bf{\Sigma}^{-1} \parcurvo{
\bf{\barra{x}} - \bf{\mu}
} + \lambda \parcurvo{\de\bf{\mu}}^\top\bf{A}^\top\\
&= n (\de \bf{\mu})^\top \parcerrado{
    \bf{\Sigma}^{-1} 
\bf{\barra{x}} - \bf{\Sigma}^{-1}\bf{\mu}}
 + \lambda \parcurvo{\de\bf{\mu}}^\top\bf{A}^\top\\
 &= (\de \bf{\mu})^\top\parllave{
    n\parcurvo{\parcerrado{
    \bf{\Sigma}^{-1} 
\bf{\barra{x}} - \bf{\Sigma}^{-1}\bf{\mu}}} + \lambda \bf{A}^\top
 }\\
 &= \parcurvo{\de \bf{\mu}}^\top \parllave{
    n\bf{\Sigma}^{-1}\barra{\bf{x}} + \bf{A}^\top\lambda - n\bf{\Sigma}^{-1}\bf{\mu}
 }\\
 &= \bf{\Sigma}^{-1}\parcurvo{\de \bf{\mu}}^\top n \parllave{
 \barra{\bf{x}} + \dfrac{\lambda}{n}\bf{\Sigma}\bf{A}^\top - \bf{\mu}
 }
\end{align*}
Igualando a 0 con la finalidad de obtener el óptimo tenemos que:
\begin{align*}
\de\mathcal{L}(\bf{\mu}) = 0 &\ssi \bf{\Sigma}^{-1}\parcurvo{\de \bf{\mu}}^\top n \parllave{
 \barra{\bf{x}} + \dfrac{\lambda}{n}\bf{\Sigma}\bf{A}^\top - \bf{\mu}
 }=0\\
  &\ssi \barra{\bf{x}} + \dfrac{\lambda}{n}\bf{\Sigma}\bf{A}^\top - \bf{\mu} = 0 \quad \estrella \\
     &\ssi \bf{\mu} = \barra{\bf{x}} + \dfrac{\lambda}{n}\bf{\Sigma}\bf{A}^\top\\
     &\ssi \bf{A}\bf{\mu} = \bf{A}
  \barra{\bf{x}} + \dfrac{\lambda}{n}\bf{A}\bf{\Sigma}\bf{A}^\top\\
    &\ssi n\parcurvo{\bf{a} -\bf{A}
  \barra{\bf{x}}}=  \lambda\parcurvo{\bf{A}\bf{\Sigma}\bf{A}^\top}\\
    &\ssi \lambda = n\parcurvo{\bf{A}\bf{\Sigma}\bf{A}^\top}^{-1}\parcurvo{\bf{a} -\bf{A}
  \barra{\bf{x}}}\\
  &\ssi \lambda = n\parcurvo{\bf{A}\bf{\Sigma}\bf{A}^\top}^{-1}\parcurvo{\bf{a} -\bf{A}
  \barra{\bf{x}}}
\end{align*}
Reemplazando en $\estrella$ tenemos que:
$$
\barra{\bf{x}} + \dfrac{\lambda}{n}\bf{\Sigma}\bf{A}^\top - \bf{\mu} = 0 \ssi \bf{\mu} = \barra{\bf{x}} + \bf{\Sigma}\bf{A}^\top\parcurvo{\bf{A}\bf{\Sigma}\bf{A}^\top}^{-1}\parcurvo{\bf{a} -\bf{A}
  \barra{\bf{x}}}
$$
Por tanto $\gorro{\bf{\mu}}_{MV}=\barra{\bf{x}} + \bf{\Sigma}\bf{A}^\top\parcurvo{\bf{A}\bf{\Sigma}\bf{A}^\top}^{-1}\parcurvo{\bf{a} -\bf{A}
  \barra{\bf{x}}}$.
  \item[\textcolor{red}{$\bf{b}$.}] Note que el Lagrangiano viene dado por:
  $$
\mathcal{L}(\bf{\mu},\bf{\Sigma}) = -\frac{np}{2}\log(2\pi)-\frac{n}{2} \log |\boldsymbol{\Sigma}|-\frac{1}{2} \operatorname{tr}\left( \boldsymbol{\Sigma}^{-1}\suma \parcurvo{
\bf{x}_i - \bf{\mu}
}\parcurvo{
\bf{x}_i - \bf{\mu}
}^\top\right) + \lambda \parcerrado{
 \bf{\mu}^\top \bf{A}^T 
}
$$
Note que al derivar el Lagrangiano con respecto a $\bf{\Sigma}$ e igualarlo a 0 se obtiene el mismo sistema que en el problema $\textcolor{red}{1.}$ por tanto se obtiene de forma directa que $\gorro{\bf{\Sigma}} = \bf{S}_{MV}$. Además si derivamos el Lagrangiano con respecto a $\bf{\mu}$ obtenemos el mismo sistema que en $\textcolor{red}{4a.}$ por tanto reemplazando el estimador de $\bf{\Sigma}$ en esta expresión obtenemos que:
$$
\gorro{\mu}_{MV} = \barra{\bf{x}} + \bf{S}_{MV}\bf{A}^\top\parcurvo{\bf{A}\bf{S}_{MV}\bf{A}^\top}^{-1}\parcurvo{\bf{a} -\bf{A}
  \barra{\bf{x}}}
$$
\end{itemize}